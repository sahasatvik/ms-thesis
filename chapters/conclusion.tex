

In conclusion, we have reviewed multiple realizations of depth functions in
the multivariate and functional settings, and seen numerous applications,
demonstrating the efficacy and versatility of this notion.
For a majority of these tasks, we have performed simulation studies,
reproduced results, and attempted to compare and contrast different procedures
proposed in the literature.
We have also tried to draw connections between different ideas; for instance,
the centrality-stability and MO-VO diagrams for functional outlier detection
(Section~\ref{sec:functional_outlier}) largely produce the same results, with
similar aims but differing approaches.
The use of random projections in functional data classification
(Section~\ref{sec:functional_classification_RP}) draws inspiration from the
random Tukey depth \parencite{albertos-reyes-2008a}, the $J$-th order depths
\parencite{nagy-gijbels-hlubinka-2017}, and the Cramer-Wold device in Hilbert
spaces \parencite{albertos-fraiman-ransford-2007}.
We have also attempted to propose a new method of regression
(Section~\ref{sec:localdepth_regression}) using the local depth regions of
\textcite{paindaveine-bever-2013}; further study into its theoretical
consistency as well as results from simulations are merited.

One area which we have not covered in much detail is regression.
For instance, \textcite{zuo-2021} gives an overview of depths used in the
context of linear regression.
\textcite{chowdhury-chaudhuri-2019} explore quantile regression for functional
data using spatial depth.
This uses the language of spatial quantiles
\parencite{chakraborty-chaudhuri-2014b}, where the spatial $\vu$-quantile
$\bm{Q}(\vu)$ for $\vu \in B^*(\bm{0}, 1) \subset \mathscr{X}^*$ is the
minimizer of
\begin{equation}
    \E_{\vX \sim F}\left[\norm{\bm{Q} - \vX} - \norm{\vX}\right] - \vu(\bm{Q}).
\end{equation}

Another aspect of depth which deserves a closer look is the Monge-Kantorovich
depth \parencite{chernozhukov-galichon-hallin-henry-2017}.
The connection between optimal transport theory and statistics is fairly new
and exciting area for exploration, and has lead to many applications even in
fields like financial mathematics, image processing, and machine learning.

The code used to produce this document, its figures, and results can be found
in this thesis' GitHub
repository\footnote{\url{https://github.com/sahasatvik/ms-thesis}}.
