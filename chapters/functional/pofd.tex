\section{Partially observed functional data}

Consider the setting where the stochastic process $\vX$ of continuous
functions is not observed on the entire interval $[0, 1]$, but rather on a
random subset $O \subseteq [0, 1]$.
Then, a dataset of partially observed curves is of the form $\mathscr{D} =
\{(\vX_i, O_i)\}_{i = 1}^n$, where $\vX_i \iid F_{\vX}$, $O_i \iid Q$ where
$Q$ generates random compact subsets of $[0, 1]$, independent of $\vX_i$.
In other words, $(\vX_i, O_i) \iid F_{\vX} \times Q$.
This setup is known as the `Missing Completely at Random' assumption.

We set $\mathscr{J}(t) = \{j\colon t \in O_j\}$ to keep track of which curves
$\vX_i$ have been observed at time $t$.
Furthermore, denote $q(t) = |\mathscr{J}(t)|$ as the number of curves $\vX_i$
observed at time $t$.

\textcite{elias-jimenez-paganoni-sangalli-2023} propose the following.

\begin{definition}[Partially observed integrated functional depth]
    Let $D$ be a $d$-variate depth function.
    The Partially Observed Integrated Functional Depth (POIFD) is defined as
    \begin{equation}
        D_{POIFD}((\vx, o), F_{\vX} \times Q) = \int_o D(\vx(t), F_{\vX(t)})\:w_o(t) \:dt,
    \end{equation}
    where $w_o(t) = q(t) / \int_o q(t)\:dt$.
\end{definition}

We can now proceed with tasks such as classification, outlier detection, etc.\
on our partially observed dataset, via depth based procedures using POIFD
values.
Another natural problem is one of curve reconstruction: given a partially
observed curve $(\vX, O)$, can we estimate $\vX$ on $M = [0, 1]\setminus O$?
