\section{Classification}

Observe that the classification procedures for multivariate data described in
Section~\ref{sec:multivariate_classification} (the maximum depth classifier,
the DD classifier, and the DD$^G$ classifier) only depend on the data through
the depth feature vectors
\begin{equation}
    \vx^D = (D(\vx, F_1), \dots, D(\vx, F_k)) \in \R^k.
\end{equation}
By simply choosing an appropriate functional data depth $D$, all of these
classification procedures naturally generalize to the functional setting.




\begin{definition}
    Let $\vX$ be a $d$-variate stochastic process of continuous functions.
    At each time point $t \in [0, 1]$, the directional outlyingness is defined
    as
    \begin{equation}
        \vO(t) = \vO(\vX(t), F_{\vX(t)}) = \left(\frac{1}{D(\vX(t), F_{\vX(t)})} - 1\right)\, \vv(t),
    \end{equation}
    where $\vv(t)$ is the unit vector pointing from the median of $F_{\vX(t)}$
    to $\vX(t)$.
\end{definition}

\begin{definition}
    The functional directional outlyingness is defined as
    \begin{equation}
        \FO(\vX, F_{\vX}) = \int_{[0, 1]} \norm{\vO(t)}^2 \:w(t)\:dt.
    \end{equation}
\end{definition}

\begin{definition}
    The mean directional outlyingness is defined as
    \begin{equation}
        \MO(\vX, F_{\vX}) = \int_{[0, 1]} \vO(t) \:w(t)\:dt.
    \end{equation}
\end{definition}

\begin{definition}
    The variation of directional outlyingness is defined as
    \begin{equation}
        \VO(\vX, F_{\vX}) = \int_{[0, 1]} \norm{\vO(t) - \MO(t)}^2 \:w(t)\:dt.
    \end{equation}
\end{definition}

Here, $w$ is a weight function on $[0, 1]$.
In our discussion, we set $w = 1$.

It is easily verified that
\begin{equation}
    \FO^2 = \norm{\MO}^2 + \VO.
\end{equation}

\textcite{dai-genton-2018} propose using the $(d + 1)$-variate feature vectors
\begin{equation}
    \vY(\vX, F_{\vX}) = (\MO^\top,\, \VO)^\top
\end{equation}
corresponding to the curve $\vX$ for the purposes of classification.
For instance, one may define the classifier
\begin{equation}
    \hat{\iota}(\vX) = \hat{\iota}_{D'}(\vY(\vX, F_i)) = \argmax_{1 \leq i \leq k} D'(\vY(\vX, F_i), F_{\vY(\vX, F_i)}),
\end{equation}
where $D'$ is a multivariate depth function.
This is simply a maximum depth classifier applied on the feature vectors
$\vY$.
When $D'$ is chosen to be the robust Mahalanobis depth, we have the classifier
\begin{equation}
    \hat{\iota}_{RM}(\vX) = \argmax_{1 \leq i \leq k} D_{RM}(\vY(\vX, F_i), F_{\vY(\vX, F_i)}).
\end{equation}



\begin{definition}
    The functional directional outlyingness matrix is defined as
    \begin{equation}
        \FOM(\vX, F_{\vX}) = \int_{[0, 1]} \vO(t)\,\vO(t)^\top \:w(t)\:dt.
    \end{equation}
\end{definition}

\begin{definition}
    The functional directional outlyingness matrix is defined as
    \begin{equation}
        \VOM(\vX, F_{\vX}) = \int_{[0, 1]} (\vO(t) - \MO(t))\,(\vO(t) - \MO(t))^\top \:w(t)\:dt.
    \end{equation}
\end{definition}

Again, it is easily verified that
\begin{equation}
    \FOM = \MO\,\MO^\top + \VOM,
\end{equation}
and that
\begin{equation}
    \FO = \tr(\FOM), \qquad
    \VO = \tr(\VOM).
\end{equation}

We may also use the feature matrix $\VOM$, or its matrix norm $\norm{\VOM}_F$
corresponding to the curve $\vX$ for the purposes of classification.
Here, $\norm{\Cdot}_F$ denotes the Frobenius norm.
For instance, a $\VOM$ based classifier may be defined as
\begin{equation}
    \hat{\iota}_{\VOM}(\vX) = \argmin_{1 \leq i \leq k} \norm{\VOM(\vX, F_i)}_F.
\end{equation}



Another approach is to use a feature vector consisting of multiple projections
of $\vX$.
Given functions $\vv_1, \dots, \vv_d$ chosen at random, we examine the
$d$-variate feature vectors
\begin{equation}
    \vZ(\vX) = \left(\ip{\vv_1}{\vX}, \dots, \ip{\vv_d}{\vX}\right)
\end{equation}
and apply a depth based multivariate classifier.
