\section{Depth contours}

The following definitions are adapted from \cite{liu-parelius-singh-1999}.

\begin{definition}
    The contour of depth $t$ is the set $\{\vx \in \R^d : D(\vx, F) = t\}$.
\end{definition}

\begin{definition}
    The region enclosed by the contour of depth $t$ is the set
    \begin{equation}
        R_F(t) \,=\, \{\vx \in \R^d : D(\vx, F) > t\}.
    \end{equation}
\end{definition}

\begin{definition}
    The $p$-th central region is the set
    \begin{equation}
        C_F(p) \,=\, \bigcap_{t}\; \{R_F(t) : P_F(R_F(t)) \geq p\}.
    \end{equation}
\end{definition}

\begin{definition}
    The $p$-th level contour, or center-outward contour surface, is the set
    $Q_F(p) = \partial C_F(p)$.
\end{definition}


\begin{example}
    Consider $\UU(B^d)$, i.e.\ the uniform distribution on the unit ball in
    $\R^d$.
    While there are no proper density contours to speak of, halfspace depth
    contours are concentric spheres centered at the origin, the deepest point.
    This illustrates how depth contours are more suited to indicating
    centrality than density contours.
\end{example}


