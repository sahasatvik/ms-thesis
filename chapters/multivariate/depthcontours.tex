\section{Depth contours}
\label{sec:multivariate_depthcontours}

Given a depth function $D$ and some fixed distribution $F \in \mathscr{F}$, we
may examine contours produced by $D(\Cdot, F)$.
The following definitions are adapted from \cite{liu-parelius-singh-1999}.

\begin{definition}
    The contour of depth $t$ is the set $\{\vx \in \R^d : D(\vx, F) = t\}$.
\end{definition}

\begin{definition}
    The region enclosed by the contour of depth $t$ is the set
    \begin{equation}
        R_F(t) \,=\, \{\vx \in \R^d : D(\vx, F) > t\}.
    \end{equation}
\end{definition}

It is often more convenient to deal with depth contours and regions based on
their probability content rather than a depth cutoff.

\begin{definition}
    The $p$-th central region is the set
    \begin{equation}
        C_F(p) \,=\, \bigcap_{t}\; \{R_F(t) : P_F(R_F(t)) \geq p\}.
    \end{equation}
\end{definition}

\begin{definition}
    The $p$-th level contour, or center-outward contour surface, is the set
    $Q_F(p) = \partial C_F(p)$.
\end{definition}


\begin{example} \label{ex:uniform}
    Consider $\UU(B^d)$, i.e.\ the uniform distribution on the unit ball in
    $\R^d$.
    While there are no proper density contours to speak of, halfspace depth
    contours are concentric spheres centered at the origin, the deepest point.
    This illustrates how depth contours are more suited to indicating
    centrality than density contours.
\end{example}


Depth based central regions and contours may be approximated empirically as
follows.

\begin{definition}
    Let $\vX_1, \dots, \vX_n \iid F$.
    We introduce depth based order statistics $\vX_{[1]}, \dots, \vX_{[n]}$,
    which are a reordering of the sample in decreasing order of depth, i.e.\
    $D(\vX_{[1]}, F) \geq \dots \geq D(\vX_{[n]}, F)$.
\end{definition}

With this, given $\vX_1, \dots, \vX_n \iid F$, the sample $p$-th central
region is given by
\begin{equation}
    C_{\hat{F}_n}(p) = \conv(\vX_{[1]}, \dots, \vX_{[\lceil np \rceil]}).
\end{equation}

The consistency of these sample central regions typically requires some
continuity of type \textbf{C3} \parencite{liu-1990, donoho-gasko-1992,
he-wang-1997}.

Depths such as the halfspace depth, the Mahalanobis depth, and the Oja depth
produce convex central regions.
Any depth satisfying \textbf{P3} produces star-shaped central regions.
Notably, the spatial depth does not necessarily produce convex nor star-shaped
central regions \parencite{nagy-2017}.


\subsection{The Monge-Kantorovich depth}

Depths which produce convex, nested central regions are appropriate for a
large class of unimodal distributions with some degree of symmetry.
Indeed, the fact that depths satisfying \textbf{P1} and \textbf{C1}
characterize elliptical distributions follows from the fact that the depth
contours coincide with density contours.
However, there are instances where this is unsuitable, such as in the
distributions illustrated in Figures~\ref{fig:localdepth_banana} and
\ref{fig:localdepth_bimodal}.
In Chapter~\ref{chap:localdepth}, we will examine the idea of \emph{local
depth}, which is capable of producing non-convex, non-star-shaped, un-nested
central regions.

Here, we briefly look at the Monge-Kantorovich depth introduced by
\textcite{chernozhukov-galichon-hallin-henry-2017}, which is also capable of
producing non-convex contours but retains their nestedness.
This is based on the idea of `transporting' contours from a reference
distribution $U_d$, say $\mathcal{U}(B^d)$, to the target distribution $F$ on
$\R^d$ via a canonical \emph{vector quantile map} $Q\colon B^d \to \R^d$.
We say that $Q$ `pushes forward' $U_d$ into $F$, denoted $Q\#F = U_d$; for $U
\sim U_d$, we have $Q(U) \sim F$.
The inverse map from $F$ to the reference $U_d$ is called the \emph{vector
rank map} $R$; we write $R\#F = U_d$.
Now, $Q$ is defined via the theory of optimal transport
\parencite{villani-2003}; it is the map which minimizes the quadratic cost
$\E_{U \sim U_d}[(Q(U) - U)^2]$ subject to $Q(U) \sim F$.
Under certain conditions, the maps $Q, R$ exist and are unique; for instance,
the Brenier-McCann theorem requires the absolute continuity of $U, F$
supported within convex subsets of $\R^d$.
With this, we supply a loose definition of the Monge-Kantorovich depth below.

\begin{definition}[Monge-Kantorovich depth]
    Let $Q$ be the vector quantile map associated with $F$, and let $R$ be its
    inverse, so that $R\#F = U_d$.
    The Monge-Kantorovich depth is defined as
    \begin{equation}
        D_{MK}(\vx, F) = D_H(R(\vx), U_d).
    \end{equation}
\end{definition}

Similarly, the Monge-Kantorovich rank of $\vx \in \R^d$ in $F$ is given by
$\norm{R(\vx)}$.
If $K_p = \partial C_{U_d}(p)$ is the $p$-th level contour of $U_d$, then the
Monge-Kantorovich $p$-quantile of $F$ is the image $Q(K_p)$.
The reference distribution $U_d$ and depth $D_H$ may of course be replaced as
necessary.

One major strength of this notion of depth is the distribution-free nature of
the ranks produced.
This has produced applications in areas such as distribution-free
(nonparametric, multivariate) testing \parencite{ghosal-sen-2022,
deb-sen-2023}.
