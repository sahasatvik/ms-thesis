Consider a class of functions $\mathscr{X}$ of the form $\vx\colon [0, 1] \to
\R^d$, equipped with a norm $\norm{\Cdot}$, and let $\mathscr{F}$ be a
suitable class of distributions on $\mathscr{X}$.
Typically, we choose $\mathscr{X}$ to be either $L_2[0, 1]$ or $\mathcal{C}[0,
1]$.
It is desirable for a depth function $D\colon \mathscr{X} \times \mathscr{F}
\to \R$ to satisfy the following properties \parencite{gijbels-nagy-2017}.

\begin{enumerate}
    \item[\textbf{P0}.] \emph{Non-degeneracy.}
    For $F \in \mathscr{F}$,
    \begin{equation}
        \inf_{\vx \in \mathscr{X}} D(\vx, F) \,<\,
        \sup_{\vx \in \mathscr{X}} D(\vx, F).
    \end{equation}
\end{enumerate}

The property \textbf{P0} has been introduced to emphasize that generalizing
multivariate depth to the functional setting requires some care; the natural
functional analogue to halfspace/Tukey depth when $\mathscr{X}$ is a Banach
space,
\begin{equation}
    D_H(\vx, F) = \inf_{\vv \in \mathscr{X}^*} P_{\vX \sim F}(\vv(\vX) \leq \vv(\vx)),
\end{equation}
turns out to be degenerate for a wide class of distributions $\mathscr{F}$
\parencite{chakraborty-chaudhuri-2014a}.
For example, when $\mathscr{X} = \mathcal{C}[0, 1]$ with the supremum norm and
$\vX$ is a Gaussian process with a positive definite covariance kernel, we
have $D_H(\Cdot, F_{\vX}) = 0$ almost surely.
A similar result holds for the analogue to the projection depth.
However, neither the functional random Tukey depth nor the functional spatial
depth
\begin{equation}
    D_S(\vx, F_{\vX}) = 1 - \left\Vert \E_{\vX \sim F} \left[\frac{\vx - \vX\;\,}{\norm{\vx - \vX}_2}\right] \right\Vert_2
\end{equation}
suffer this deficiency when $\mathscr{X} = L_2$
\parencite{albertos-reyes-2008a, gijbels-nagy-2017}.


The remaining properties are analogues of the Zuo-Serfling properties for
multivariate depth functions.
First, the  notion of affine invariance in \textbf{P1} can be generalized in
many ways; \textcite{gijbels-nagy-2017} recommend the following.

\begin{enumerate}
    \item[\textbf{P1S}.] \emph{Scalar-affine invariance.}
    For $a, b \in \R$ with $a$ non-zero and $\vx \in \mathscr{X}$,
    \begin{equation}
        D(a\vx + b, F_{a\vX + b}) = D(\vx, F_{\vX}).
    \end{equation}

    \item[\textbf{P1F}.] \emph{Function-affine invariance.}
    For $a, b, x \in \mathscr{X}$ with $ax \in \mathscr{X}$,
    \begin{equation}
        D(ax + b, F_{aX + b}) = D(x, F_{X}).
    \end{equation}
\end{enumerate}


When generalizing \textbf{P2}, we must first define a notion of symmetry of $F
\in \mathscr{F}$.
To this end, we say that $F_{\vX}$ is symmetric about $\vth \in \mathscr{X}$
if for all $\varphi \in \mathscr{X}^*$, we have $\varphi(\vX)$ is symmetric
about $\varphi(\vth)$.
Again, we are free to choose our notion of univariate symmetry for
$\varphi(\vX)$.
\textcite{gijbels-nagy-2017} consider central and halfspace symmetry.

\begin{enumerate}
    \item[\textbf{P2C}.] \emph{Maximality at center of central symmetry.}
    Any centrally symmetric $F \in \mathscr{F}$ is symmetric about $\vth \in
    \mathscr{X}$ if and only if $D(\vth, F) = \sup_{\vx \in \mathscr{X}}
    D(\vx, F)$.

    \item[\textbf{P2H}.] \emph{Maximality at center of halfspace symmetry.}
    Any halfspace symmetric $F \in \mathscr{F}$ is symmetric about $\vth \in
    \mathscr{X}$ if and only if $D(\vth, F) = \sup_{\vx \in \mathscr{X}}
    D(\vx, F)$.
\end{enumerate}

Earlier, \textcite{reyes-battey-2016} proposed the following variant of
\textbf{P2}.
\begin{enumerate}
    \item[\textbf{P2G}.] \emph{Maximality at Gaussian process mean.}
    For a zero-mean, stationary, almost surely continuous Gaussian process $F
    \in \mathscr{F}$, we have $D(\vth, F) = \sup_{\vx \in \mathscr{X}} D(\vx,
    F)$ where $\theta$ is the zero mean function.
\end{enumerate}
The above notions of maximality at the center are \textbf{P2H} $>$
\textbf{P2C} $>$ \textbf{P2G} in order of strength.

The properties \textbf{P3} and \textbf{P4} have straightforward
generalizations.
\begin{enumerate}
    \item[\textbf{P3D}.] \emph{Monotonicity relative to deepest point.}
    For $F \in \mathscr{F}$ such that $D(\vth, F) = \sup_{\vx \in \mathscr{X}}
    D(\vx, F)$, we have for $\alpha \in [0, 1]$,
    \begin{equation}
        D(\vx, F) \leq D(\vth + \alpha(\vx - \vth), F).
    \end{equation}

    \item[\textbf{P4V}.] \emph{Vanishing at infinity.} For any $F \in
    \mathscr{F}$,
    \begin{equation}
        D(\vx, F) \to 0\;\text{ as }\; \norm{\vx} \to \infty.
    \end{equation}
\end{enumerate}


\textcite{reyes-battey-2016} and \textcite{gijbels-nagy-2017} also deal with
the notions of continuity in $F$.
Let $d_{\mathscr{F}}$ metrize the topology of weak convergence in
$\mathscr{F}$.
\begin{enumerate}
    \item[\textbf{C2W}.] \emph{Weak continuity in $F$.}
    For all $\epsilon > 0$ and $F \in \mathscr{F}$, there exists $\delta > 0$
    such that for all $G \in \mathscr{F}$ such that $d_{\mathscr{F}}(F, G) <
    \delta$, we have $|D(\vx, F) - D(\vx, G)| < \epsilon$, $F$-almost surely.

    \item[\textbf{C2U}.] \emph{Uniform continuity in $F$.}
    For all $\epsilon > 0$ and $F \in \mathscr{F}$, there exists $\delta > 0$
    such that for all $G \in \mathscr{F}$ such that $d_{\mathscr{F}}(F, G) <
    \delta$, we have $\sup_{\vx \in \mathscr{X}} |D(\vx, F) - D(\vx, G)| <
    \epsilon$.
\end{enumerate}


\textcite[Table 1]{gijbels-nagy-2017} provides a detailed summary of which of
these properties are satisfied by the depth functions discussed in the
following section.


\section{Functional depth functions}

\subsection{Summary depths}

Let $D$ be a univariate or multivariate depth function.
We can use this to define the depth of a curve $\vx$ by first computing the
multivariate $D$-depth of each time slice $\vx(t)$, then `summarizing' these
depths over all $t \in [0, 1]$.
One possibility is to take a simple or weighted time average, as in the
integrated depth \parencite{fraiman-muniz-2001}.

\begin{definition}[Fraiman-Muniz depth]
    \label{def:FM_depth}
    The integrated depth, or Fraiman-Muniz depth, is defined as
    \begin{equation}
        D_F(\vx, F_{\vX}) = \int_{[0, 1]} D(\vx(t), F_{\vX(t)})\:w(t)\:dt.
    \end{equation}
    Here, $w$ is a weight function.
\end{definition}

Alternatively, we may choose the lowest or `worst' depth over time
\parencite{mosler-2013}.
This way, low depth values over small portions of time, which indicate a
deviation from centrality, are better reflected in the summary.

\begin{definition}[Infimal depth]
    The infimal depth is defined as
    \begin{equation}
        D_{Inf}(\vx, F_{\vX}) = \inf_{t \in [0, 1]} D(\vx(t), F_{\vX(t)}).
    \end{equation}
\end{definition}


\textcite{nagy-gijbels-hlubinka-2017}, motivated by the problem of detecting
\emph{shape outliers}, extend the definitions of Fraiman-Muniz depth and
infimal depth as follows.
We will examine their significance briefly in
Section~\ref{sec:functional_outlier}.

\begin{definition} \label{def:J_FM_depth}
    The $J$-th order integrated depth is defined as
    \begin{equation}
        D_F^J(\vx, F_{\vX}) = \int_{[0, 1]^J} D((\vx(t_1), \dots, \vx(t_J))^\top, F_{(\vX(t_1), \dots, \vX(t_J))^\top}) \:w(\bm{t})\:d\bm{t}.
    \end{equation}
\end{definition}

\begin{definition} \label{def:J_Inf_depth}
    The $J$-th order infimal depth is defined as
    \begin{equation}
        D_{Inf}^J(\vx, F_{\vX}) = \inf_{\bm{t} \in [0, 1]^J} D((\vx(t_1), \dots, \vx(t_J))^\top, F_{(\vX(t_1), \dots, \vX(t_J))^\top}.
    \end{equation}
\end{definition}

\begin{remark}
    It is often convenient to use Monte-Carlo approximations of the $J$-th
    order Fraiman-Muniz and infimal depths.
\end{remark}



\subsection{Band depths}

\textcite{pintado-romo-2009} later introduced the notion of band depth for
univariate functional data.

\begin{definition}[Band depth]
    The band depth, for some index $J \geq 2$, is defined as
    \begin{equation}
        D_B^J(\vx, F_{\vX}) = \sum_{j = 2}^J\, P_{\vX_i \iid F_{\vX}}(\vx \in \conv(\vX_1, \dots, \vX_j)).
    \end{equation}
\end{definition}
The empirical version of band depth is defined as
\begin{equation}
    D_B^J(\vx, \hat{F}_n) = \sum_{j = 2}^J\binom{n}{j}^{-1} \hspace{-1em}\sum_{\substack{1 \leq i_1 < \dots < i_j \leq n}} \bm{1}(\vx \in \conv(\vx_{i_1}, \dots, \vx_{i_j})).
\end{equation}
This is simply the proportion of $j$-tuples of curves (for $2 \leq j \leq J$)
which envelope $\vx$.
Note that if two curves intersect at a point, a third curve is enveloped by
them only when it passes through the point of intersection.
For most commonly used $F_{\vX}$, this happens with probability zero, making
the band depth for $J = 2$ degenerate.
Thus, we generally use $J = 3$.


\begin{remark}
    The band depth may fail to satisfy \textbf{P0} even for $J \geq 3$.
    It follows from \textcite[Theorem~3.2]{chakraborty-chaudhuri-2014a} that
    when $\mathscr{X} = \mathcal{C}[0, 1]$ and $\vX$ is a Feller process (for
    instance, Brownian motion) such that $P(X_0 = 0) = 1$ and the distribution
    of each $\vX_t$ for $t \in (0, 1]$ is non-atomic and symmetric about $0$,
    the band depth $D^J_B(\Cdot, F_{\vX}) = 0$ almost surely.
    The following modification of the band depth resolves this issue.
\end{remark}


\begin{definition}[Modified band depth]
    Define the enveloping time
    \begin{equation}
        \ET(\vx;\, \vx_1, \dots, \vx_j) = m_1(\{t \in [0, 1]\colon \vx \in \conv(\vx_1, \dots, \vx_j)\}),
    \end{equation}
    where $m_1$ is the Lebesgue measure on $\R$.
    The modified band depth is defined as
    \begin{equation}
        D_{MB}^J(\vx, F_{\vX}) = \sum_{j = 2}^J\, \E_{\vX_i \iid F_{\vX}}\left[ \ET(\vx;\, \vX_1, \dots, \vX_j)\right].
    \end{equation}
\end{definition}
The empirical version of modified band depth is defined as
\begin{equation}
    D_{MB}^J(\vx, \hat{F}_n) = \sum_{j = 2}^J\binom{n}{j}^{-1} \hspace{-1em} \sum_{\substack{1 \leq i_1 < \dots < i_j \leq n}} \ET(\vx;\, \vx_{i_1}, \dots, \vx_{i_j}).
\end{equation}
We generally use $J = 2$ for ease of computation, and denote the corresponding
modified band depth simply as $D_{MB}(\Cdot, \Cdot)$, dropping the
superscript.
\begin{equation}
    D_{MB}(\vx, \hat{F}_n) = \binom{n}{2}^{-1}\sum_{i = 1}^n\sum_{j = i + 1}^n \ET(\vx; \vx_i, \vx_j).
\end{equation}



\subsection{Half-region depths}

Later, \textcite{pintado-romo-2011} introduced the half-region depth.

\begin{definition}
    We say that $\vy$ is in the hypograph of $\vx$, denoted, $\vy \in
    H_{\vx}$, if $\vy(t) \leq \vx(t)$ for all $t \in [0, 1]$.
    Similarly, we say that $\vy$ is in the epigraph of $\vx$, denoted, $\vy
    \in E_{\vx}$, if $\vy(t) \geq \vx(t)$ for all $t \in [0, 1]$.
\end{definition}

\begin{definition}[Half-region depth]
    The half-region depth is defined as
    \begin{equation}
        D_{HR}(\vx, F) = \min\{P_F(H_{\vx}),\, P_F(E_{\vx})\}.
    \end{equation}
\end{definition}

The quantity $P_F(E_{\vx})$ is called the epigraph index, which measures the
proportion of curves that lie entirely above $\vx$.

\begin{remark}
    The half-region depth may also fail to satisfy \textbf{P0}, with the same
    counterexample used earlier for the degeneracy of the band depth
    \parencite[Theorem~3.2]{chakraborty-chaudhuri-2014a}.
\end{remark}

\begin{definition}[Modified half-region depth]
    Denote the superior modified hypograph (MHI) and epigraph (MEI) indices
    \begin{align}
        \MHI_F(\vx) &= \E_{\vX \sim F}[m_1(\{t \in [0, 1]\colon \vx(t) \geq \vX(t)\})], \\
        \MEI_F(\vx) &= \E_{\vX \sim F}[m_1(\{t \in [0, 1]\colon \vx(t) \leq \vX(t)\})].
    \end{align}
    The modified half-region depth is defined as
    \begin{equation}
        D_{MHR}(\vx, F) = \min\{\MHI_F(\vx),\, \MEI_F(\vx)\}.
    \end{equation}
\end{definition}


\section{Classification}

Observe that the classification procedures for multivariate data described in
Section~\ref{sec:multivariate_classification} (the maximum depth classifier,
the DD classifier, and the DD$^G$ classifier) only depend on the data through
the depth feature vectors
\begin{equation}
    \vx^D = (D(\vx, F_1), \dots, D(\vx, F_k)) \in \R^k.
\end{equation}
By simply choosing an appropriate functional data depth $D$, all of these
classification procedures naturally generalize to the functional setting.




\textcite{dai-genton-2018} proposed a method which measures the outlyingness
of $\vx$ with respect to a population via depth as follows.

\begin{definition}
    Let $\vX$ be a $d$-variate stochastic process of continuous functions.
    At each time point $t \in [0, 1]$, the directional outlyingness is defined
    as
    \begin{equation}
        \vO(t) = \vO(\vX(t), F_{\vX(t)}) = \left(\frac{1}{D(\vX(t), F_{\vX(t)})} - 1\right)\, \vv(t),
    \end{equation}
    where $\vv(t)$ is the unit vector pointing from the median of $F_{\vX(t)}$
    to $\vX(t)$.
\end{definition}

\begin{definition}
    The functional directional outlyingness is defined as
    \begin{equation}
        \FO(\vX, F_{\vX}) = \int_{[0, 1]} \norm{\vO(t)}^2 \:w(t)\:dt.
    \end{equation}
\end{definition}

\begin{definition}
    The mean directional outlyingness is defined as
    \begin{equation}
        \MO(\vX, F_{\vX}) = \int_{[0, 1]} \vO(t) \:w(t)\:dt.
    \end{equation}
\end{definition}

\begin{definition}
    The variation of directional outlyingness is defined as
    \begin{equation}
        \VO(\vX, F_{\vX}) = \int_{[0, 1]} \norm{\vO(t) - \MO(t)}^2 \:w(t)\:dt.
    \end{equation}
\end{definition}

Here, $w$ is a weight function on $[0, 1]$.
In our discussion, we set $w = 1$.

It is easily verified that
\begin{equation}
    \FO^2 = \norm{\MO}^2 + \VO.
\end{equation}

\textcite{dai-genton-2018} propose using the $(d + 1)$-variate feature vectors
\begin{equation}
    \vY(\vX, F_{\vX}) = (\MO^\top,\, \VO)^\top
\end{equation}
corresponding to the curve $\vX$ for the purposes of classification.
For instance, one may define the classifier
\begin{equation}
    \hat{\iota}(\vX) = \hat{\iota}_{D'}(\vY(\vX, F_i)) = \argmax_{1 \leq i \leq k} D'(\vY(\vX, F_i), F_{\vY(\vX, F_i)}),
\end{equation}
where $D'$ is a multivariate depth function.
This is simply a maximum depth classifier applied on the feature vectors
$\vY$.
When $D'$ is chosen to be the robust Mahalanobis depth, we have the classifier
\begin{equation}
    \hat{\iota}_{RM}(\vX) = \argmax_{1 \leq i \leq k} D_{RM}(\vY(\vX, F_i), F_{\vY(\vX, F_i)}).
\end{equation}


\begin{definition}
    The functional directional outlyingness matrix is defined as
    \begin{equation}
        \FOM(\vX, F_{\vX}) = \int_{[0, 1]} \vO(t)\,\vO(t)^\top \:w(t)\:dt.
    \end{equation}
\end{definition}

\begin{definition}
    The functional directional outlyingness matrix is defined as
    \begin{equation}
        \VOM(\vX, F_{\vX}) = \int_{[0, 1]} (\vO(t) - \MO(t))\,(\vO(t) - \MO(t))^\top \:w(t)\:dt.
    \end{equation}
\end{definition}

Again, it is easily verified that
\begin{equation}
    \FOM = \MO\,\MO^\top + \VOM,
\end{equation}
and that
\begin{equation}
    \FO = \tr(\FOM), \qquad
    \VO = \tr(\VOM).
\end{equation}

We may also use the feature matrix $\VOM$, or its matrix norm $\norm{\VOM}_F$
corresponding to the curve $\vX$ for the purposes of classification.
Here, $\norm{\Cdot}_F$ denotes the Frobenius norm.
For instance, a $\VOM$ based classifier may be defined as
\begin{equation}
    \hat{\iota}_{\VOM}(\vX) = \argmin_{1 \leq i \leq k} \norm{\VOM(\vX, F_i)}_F.
\end{equation}



Another approach is to use a feature vector consisting of multiple projections
of $\vX$.
Given functions $\vv_1, \dots, \vv_d$ chosen at random, we examine the
$d$-variate feature vectors
\begin{equation}
    \vZ(\vX) = \left(\ip{\vv_1}{\vX}, \dots, \ip{\vv_d}{\vX}\right)
\end{equation}
and apply a depth based multivariate classifier.


\section{Outlier detection}

\begin{figure}
    \centering
    \includegraphics[width = \textwidth]{outliergram_wine}
    \caption{
        Outliergram for the NMR spectra of 40 wine samples.
        The three purple curves have been identified as shape outliers, as
        they fall outside the orange ribbon in the outliergram.
        Although the red curves lie on the orange parabola, they have low MBD
        and extreme MEI values, indicating that they lie above or below the
        main mass of curves.
    }
    \label{fig:outliergram_wine}
\end{figure}



\begin{figure}
    \centering
    \includegraphics[width = \textwidth]{centrality_stability_wine}
    \caption{
        Centrality-stability diagram for the NMR spectra of 40 wine samples.
        The red curves are seen to deviate in terms of centrality, indicated
        by the fact that the corresponding points in the centrality-stability
        diagram fall towards the right.
        The purple curves deviate in terms of stability, with the green curve
        showing extreme deviation.
    }
    \label{fig:centrality_stability_wine}
\end{figure}


\begin{figure}
    \centering
    \includegraphics[width = \textwidth]{outlyingness_heatmap_wine}
    \caption{
        Outlyingness heatmap for the NMR spectra of 40 wine samples.
        The extreme curve \#37 has been zeroed out in the second diagram to
        better illustrate the variation in outlyingness for the remaining
        curves.
    }
    \label{fig:outlyingness_heatmap_wine}
\end{figure}



\begin{figure}
    \centering
    \includegraphics[width = \textwidth]{MO_VO_wine}
    \caption{
        MO-VO diagram for the NMR spectra of 40 wine samples.
    }
    \label{fig:MO_VO_wine}
\end{figure}





\begin{figure}
    \centering
    \includegraphics[width = \textwidth]{outlyingness_octane}
    \caption{
        Outliergram, centrality-stability, and MO-VO diagrams for the NIR
        spectra of 39 gasoline samples.
        The six purple curves \#25, 26, 36-39 correspond to samples containing
        added alcohol.
        While the outliergram does not clearly identify these outliers, the
        centrality-stability and MO-VO diagrams show a marked separation from
        the main curves.
    }
    \label{fig:outlyingness_octane}
\end{figure}



\begin{figure}
    \centering
    \includegraphics[width = \textwidth]{outlyingness_heatmap_octane}
    \caption{
        Outlyingness heatmap for the NIR spectra of 39 gasoline samples.
        The outlying curves have been zeroed out in the second diagram.
    }
    \label{fig:outlyingness_heatmap_octane}
\end{figure}



% \section{Clustering}

\section{Partially observed functional data}

Consider the setting where the stochastic process $\vX$ of continuous
functions is not observed on the entire interval $[0, 1]$, but rather on a
random subset $O \subseteq [0, 1]$.
Then, a dataset of partially observed curves is of the form $\mathscr{D} =
\{(\vX_i, O_i)\}_{i = 1}^n$, where $\vX_i \iid F_{\vX}$, $O_i \iid Q$ where
$Q$ generates random compact subsets of $[0, 1]$, independent of $\vX_i$.
In other words, $(\vX_i, O_i) \iid F_{\vX} \times Q$.
This setup is known as the `Missing Completely at Random' assumption.

We set $\mathscr{J}(t) = \{j\colon t \in O_j\}$ to keep track of which curves
$\vX_i$ have been observed at time $t$.
Furthermore, denote $q(t) = |\mathscr{J}(t)|$ as the number of curves $\vX_i$
observed at time $t$.

\textcite{elias-jimenez-paganoni-sangalli-2023} propose the following.

\begin{definition}[Partially observed integrated functional depth]
    Let $D$ be a $d$-variate depth function.
    The Partially Observed Integrated Functional Depth (POIFD) is defined as
    \begin{equation}
        D_{POIFD}((\vx, o), F_{\vX} \times Q) = \int_o D(\vx(t), F_{\vX(t)})\:w_o(t) \:dt,
    \end{equation}
    where $w_o(t) = q(t) / \int_o q(t)\:dt$.
\end{definition}

We can now proceed with tasks such as classification, outlier detection, etc.\
on our partially observed dataset, via depth based procedures using POIFD
values.
Another natural problem is one of curve reconstruction: given a partially
observed curve $(\vX, O)$, can we estimate $\vX$ on $M = [0, 1]\setminus O$?

